 % % % % % % % % % % % % % % % % % % % % % % % % % % % % % % % % % 
 %  
 %                       Copyright 2009 Daisuke TAKAGO
 %                   takago@neptune.kanazawa-it.ac.jp
 %
 % % % % % % % % % % % % % % % % % % % % % % % % % % % % % % % % %
 %
 %  This file is part of  "EZ8085"
 %
 %  EZ8085 is free software: you can redistribute it and/or modify
 %  it under the terms of the GNU General Public License as published by
 %  the Free Software Foundation, either version 3 of the License, or
 %  (at your option) any later version.
 %
 %  EZ8085 program is distributed in the hope that it will be useful,
 %  but WITHOUT ANY WARRANTY; without even the implied warranty of
 %  MERCHANTABILITY or FITNESS FOR A PARTICULAR PURPOSE.  See the
 %  GNU General Public License for more details.
 % 
 %  You should have received a copy of the GNU General Public License
 %  along with EZ8085.  If not, see <http://www.gnu.org/licenses/>.
 %
 % % % % % % % % % % % % % % % % % % % % % % % % % % % % % % % % %

\documentclass{jsarticle}
\pagestyle{empty}
\usepackage{okumacro,url}
%%%%%%%%%%%%%%%%%%%%%%%%
\twocolumn
\oddsidemargin=-0.4truein
\topmargin=-0.3truein
\headheight=0truein
\headsep=0truein
\textwidth=7.2truein
\textheight=10.6truein
%%%%%%%%%%%%%%%%%%%%%%%%%%%%%%%%%
\begin{document}
\begin{screen}
{\Huge\bf {\sf EZ8085取扱い説明書}}
{\footnotesize \url{http://pegasus.infor.kanazawa-it.ac.jp/~takago/ez8085}}
\end{screen}
\section{EZ8085とは}
EZ8085はUNIXシステム上で動作する8085エミュレータ・デバッガです
\footnote{割り込みやIOポート制御の命令は動作しません.}.アセンブル
\footnote{簡易アセンブラの為,アセンブルエラーは表示しません.},逆アセンブル,レジスタ・メモリの操作などの機能を備えています.改造・再配布は自由ですがGPLに従ってください.
\subsection{EZ8085のビルド,起動と終了}
\noindent
\verb| $ tar xvfz ez8085-0.15.tar.gz| \return \dotfill 解凍\\
\verb| $ cd ez8085-0.15|\return \dotfill ディレクトリ移動\\
\verb| $ make|\return \dotfill ビルド\\
\verb| $ ./ez8085|\return \dotfill EZ8085起動\\
~~~EZ8085のコンソールで,\verb|exit|\return と打つと終了します.
\section{各機能の使い方}
\subsection{簡易アセンブラを使う}
\verb|asm|\return \footnote{\verb|asm ファイル名|\return で,指定ファイルをアセ
ンブルできます.}でアセンブルモードに移り,\return でアセンブルモー
ドを抜けます.アセンブルモードでは1行毎にマシンコードに変換し,そのまま8085のメモリにロードします.
\begin{itemize}
\item {\bf セミコロン;}から行末まではコメントと見做されます.
\item 利用可能なディレクティブ\\
\verb|ORG 8000h|\dotfill ロケーションカウンタの変更\\
\verb|DB 12h,45h,DEh,DEh| \dotfill バイトデータ格納\\
\verb|DW 1234h,ABCDh| \dotfill ワードデータ格納\\
\verb|DS "Alpha",'X'| \dotfill 文字列格納\\
\item   \verb|MVI A,30h| の様に, オペコード と オペランドの間にはスペー
	スを1個だけ空けてください.また,{\bf コンマ,}の前後にはスペースを空けないで下さい.
\item ラベルを使うことができます.使用例)\\
\verb|LOOP:  LDA 0B000h|\\
\verb|       JMP LOOP|
\begin{itemize}
\item 未定義ラベルを使った場合,その
      部分はアセンブルモードから抜ける際に,絶対番地に置き換わります(最
      後に表示される{\bf シンボルテーブル}などで確認して下さい).
\item 3バイト命令のオペランドでは,頭に0をつけて数値とラベルと区別して下さい.
\end{itemize}
 \end{itemize}
\subsection{逆アセンブラを使う}
\verb|disas 8000h 10|\return とすると8000h番地から,10命令分を逆アセンブルします.

\subsection{レジスタを操作する}
\begin{itemize}
\item レジスタ値の変更は以下の様にします.\\
   \verb|PC=4000h|\return\dotfill プログラムカウンタの変更\\
   \verb|A=F4h|\return\dotfill アキュムレータの変更\\
   \verb|c=1|\return \dotfill キャリフラグのセット\\   
   \verb|F=A2h|\return\dotfill フラグレジスタをまとめて変更\\
   \verb|BC=ABCDh|\return\dotfill ペアレジスタにセット\\
   \verb|C=13h|\return\dotfill Cレジスタの変更
\item \verb|reg|\return \dotfill 全レジスタを表示
\item \verb|reset|\return \dotfill 全レジスタをリセット
\end{itemize}
\subsection{メモリを操作する}
\begin{itemize}
\item \verb|[8000h]|\return \dotfill 8000h番地の内容を表示
\item \verb|[8000h]=d2h|\return \dotfill 8000h番地にd2を格納
\item \verb|dump 4000h 10h|\return \\\dotfill 4000h番地から16バイトHEXダン
      プ
\item \verb|adump 4000h 10h|\return \\\dotfill 4000h番地から16バイトASCIIダンプ
\item \verb|load 8000h 12h,34h,deh|\return \\\dotfill 8000h番地から連続的にデータを格納
\item \verb|hexload 8000h|\return \\
\dotfill 8000h番地から連続的に16進系列を格納
\item \verb|fill 4000h 10 FFh|\return\\ \dotfill 4000h番地から16バイト分を
      FFhで埋める
\end{itemize}
\subsection{プログラムを実行する}
\begin{itemize}
\item \verb|step|\return または単に\return \dotfill {\bf シングルステップ実行}
\item
\verb|cont|\return \dotfill {\bf 連続実行}
\begin{itemize}
\item 無限ループに入っても, \keytop{Ctrl}+\keytop{c}で中断可能
\item \verb|HLT|を実行するか,{\gt テンポラリブレー
     クポイント}で連続実行は中断されます.例えば,\verb|brk 4000h|\return
      とすると,4000h番地にブレークポイントを設定すること
      ができます.ただし,一度停止するとそのポイントは消滅します.
\end{itemize}
\item  \verb|HLT|命令や未知のマシンコードを実行するとCPUが停止します.\verb|resume|\return で再開します.
\end{itemize}
\subsection{簡易ヘルプを表示する}
\verb|help|\return で簡単な(お粗末な)ヘルプを表示します.
\end{document}
